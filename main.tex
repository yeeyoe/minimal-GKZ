% 预览源文件

%% LyX 2.3.6.1 created this file.  For more info, see http://www.lyx.org/.
%% Do not edit unless you really know what you are doing.
\documentclass[oneside,english]{ctexbook}
\usepackage[T1]{fontenc}
\usepackage{CJKutf8}
\setcounter{tocdepth}{3}
\usepackage{amsmath}
\usepackage{amsthm}
\usepackage{amssymb}

\makeatletter
%%%%%%%%%%%%%%%%%%%%%%%%%%%%%% Textclass specific LaTeX commands.
\numberwithin{equation}{section}
\numberwithin{figure}{section}
\theoremstyle{plain}
\newtheorem{thm}{\protect\theoremname}
\theoremstyle{definition}
\newtheorem{xca}[thm]{\protect\exercisename}

%%%%%%%%%%%%%%%%%%%%%%%%%%%%%% User specified LaTeX commands.
\newcommand{\exercisename}{题目}

\makeatother

\usepackage{babel}
\providecommand{\exercisename}{Exercise}
\providecommand{\theoremname}{Theorem}

\begin{document}
\begin{center}
\textbf{\Large{}数学分析 3 第 1 次小测验}{\Large\par}
\par\end{center}
\begin{xca}[10分]
 把函数 $f(x)=(x-1)^{2}$ 在 $(0,1)$ 上展开成\textbf{余弦}级数, 并由收敛定理证明:
\[
\frac{\pi^{2}}{6}=1+\frac{1}{2^{2}}+\frac{1}{3^{2}}+\cdots.
\]
\end{xca}

\begin{xca}[10分]
 指出点集\textbf{$E=\{(x,\frac{1}{x}\sin\frac{1}{x})\mid x>0\}$}的\textbf{内部},\textbf{边界}和\textbf{聚点集}.
\end{xca}

\begin{xca}[20分]
 下列极限是否存在,若存在求极限。

$(1)\lim_{(x,y)\rightarrow(0,0)}(x+y)\ln\sqrt{x^{2}+y^{2}}$

$\ensuremath{(2)\lim_{(x,y)\rightarrow(0,0)}\frac{\sin xy}{x^{2}+y^{2}}}$

$(3)\lim_{(x,y)\rightarrow(+\infty,+\infty)}\frac{x^{2}+y^{2}}{x^{3}+y^{3}}$

$(4)\lim_{(x,y)\rightarrow(0,0)}\frac{x^{3}+y^{3}}{x^{2}+y}$
\end{xca}

\begin{xca}[10分]
 设函数
\[
f(x,y)=\left\{ \begin{array}{cc}
\frac{x^{3}}{x^{2}+y^{2}}, & x^{2}+y^{2}\neq0\\
0, & x^{2}+y^{2}=0
\end{array}\right.
\]
(1) 求 $f$ 在原点处的偏导数; (2) 判定 $f$ 在原点处是否可微? 
\end{xca}

\begin{xca}[20分]
求导

$(1)f(x,y)=xy\cos xy,\ \text{求 }f_{x}$

$(2)f(x,y)=x^{y},\ \text{求 }f_{x},f_{y}$ 

$(3)f(x,y)=u\left(xy^{2},x^{2}y\right),\ \text{求 }f_{xy}$ ($u$ 的偏导数请用
$u_{1},u_{2},u_{12}$ ... 表示)

$(4)f(x)=\int_{x}^{x^{2}}u(x,t)dt$, 求 $\frac{df}{dx}$
\end{xca}

\begin{xca}[10分]
 求函数 $f(x,y)=4x+xy^{2}+y^{2}$ 在区域 $\{(x,y)\mid x^{2}+y^{2}\leqslant1\}$
上的最大值和最小值。 
\end{xca}

\begin{xca}[10分]
证明: 如果 $f(x,y)$ 关于变量 $x$ 连续(当$y$固定时), 且偏导函数 $f_{y}$ 有界, 则 $f$
为二元连续函数。
\end{xca}

\begin{xca}[10分]
 设 $\Omega\subset\mathbb{R}^{2}$为有界开区域,$u(x,y)$ 在 $\Omega$ 上具有二阶连续的偏导数。

(1) 若处处有 $\triangle u=u_{xx}+u_{yy}>0$,证明 $u$ 不能在 $\Omega$ 中某点处取到最大值。

(2) 若 $u$ 在 $\bar{\Omega}=\Omega\cup\partial\Omega$ 上连续,在 $\Omega$
上处处有 $\triangle u\geq0$,证明 $\sup_{\bar{\Omega}}u=\sup_{\partial\Omega}u$,
即 $u$ 可以在边界上取到最大值。
\end{xca}


\end{document}

